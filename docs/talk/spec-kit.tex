\documentclass{beamer}

\usepackage{twemojis}



\title[Spec-Kit Demo]{AI Development With Spec-Kit:\texorpdfstring{\\\small}{ }Turning ``prompt and pray'' into an actual engineering discipline}
\author{Jeff Richley}
% \date{\small A live demo featuring pandas, penguins, and at least one otter who tried to sabotage my talk last time.}
\date{\today}


\begin{document}

% ------------------------------------------------
\begin{frame}
  \titlepage
  \vfill
  \centering
  {\small A live demo featuring pandas, penguins, and at least one otter who tried to sabotage my talk last time.}
\end{frame}

% ------------------------------------------------
\section{The Problem}

\begin{frame}{Why AI Development Feels Chaotic}
AI projects often go off the rails because:
\begin{itemize}
  \item Requirements drift faster than your attention span during a Zoom meeting
  \item All the ``context'' is scattered across Slack, GitHub, email, sticky notes, and that one brilliant thought you had in the shower
  \item The AI confidently generates code\ldots{} for a completely different problem
  \item There is absolutely \emph{not} a repeatable development cycle --- unless ``winging it'' somehow counts
\end{itemize}

\bigskip
\textbf{Goal:} Build AI software \emph{without} chaos. (Finally.)
\end{frame}


% ------------------------------------------------
\section{Introducing Spec-Kit}

\begin{frame}{What Spec-Kit Actually Does}
Spec-Kit gives you:
\begin{itemize}
  \item A workflow that gently forces your AI to act like it's seen a project plan before
  \item A clean separation between \emph{what you want}, \emph{how you want it}, and \emph{how badly the AI will try to misunderstand it}
  \item Repeatable, auditable steps --- so future-you can't yell at past-you
  \item A simple system that works with any AI coding agent (even the ones with ``creative'' tendencies)
\end{itemize}

\bigskip
\textbf{Not heavy. Not bureaucratic. Just enough structure to keep things sane.}
\end{frame}


% ------------------------------------------------
\begin{frame}{The Four Pillars (a.k.a. The Sacred Scrolls)}
    Spec-Kit consists of:
    \begin{enumerate}
      \item \textbf{/specify} --- Define \emph{what} you want and \emph{why}, before the AI goes rogue
      \item \textbf{/plan} --- Decide \emph{how} you want it built, so the AI doesn’t freestyle jazz your architecture
      \item \textbf{/tasks} --- Break the project into tiny steps the AI cannot possibly misinterpret
      \item \textbf{/implement} --- Release the AI to generate code\ldots{} but with \emph{responsible adult supervision}
    \end{enumerate}
    \end{frame}


% ------------------------------------------------
\section{Demo App Story}

\begin{frame}{The Demo: ``Pick-a-Zoo''}

\textbf{Previously on\ldots{} Pick-a-Zoo:}\\
    While planning this talk, the otters staged a small yet passionate union rally.
    They demanded more screen time, better lighting, and dental.\\
    Negotiations are\ldots{} ongoing.
    

\bigskip

\textbf{What we have so far:}
\begin{itemize}
  \item \textbf{Live-stream switcher} for real animal cams
  \item \textbf{Cast:} elephants \twemoji{1f418}, hippos \twemoji{1f99b}, otters \twemoji{1f9a6} (Local 404), penguins \twemoji{1f427}, tigers \twemoji{1f405}
  \item \textbf{Snapshot button} --- capturing ``evidence'' since day one
\end{itemize}

\bigskip

\textbf{What we're building today (live):}
\begin{itemize}
  \item \textbf{Time-lapse feature} --- otter chaos at 60× speed
\end{itemize}

\bigskip
\textbf{Cute animals + structured AI dev = today's adventure.}

\end{frame}


    
% ------------------------------------------------
\section{Spec-Kit Flow: Live Demo Breakdown}

\begin{frame}{Step 1: /specify --- The Spec Before the Chaos}
Capturing requirements clearly --- you know, like actual engineering:

\begin{itemize}
  \item \textbf{What are we building?} \\
        A real description, not ``something with AI'' scribbled on a napkin.
  \item \textbf{Why does it matter?} \\
        If you can't answer this, congratulations: you've discovered scope creep.
  \item \textbf{What does success look like?} \\
        Measurable, testable, and the opposite of ``vibes.''
\end{itemize}

\bigskip
In real engineering, specifications define behavior, constraints, and acceptance criteria. \\
\textbf{/specify does the same thing --- just without the 47-page PDF.}

\bigskip
\textbf{No tech. No code. Just clarity before the AI starts free-styling.}
\end{frame}


% ------------------------------------------------
\begin{frame}{Step 2: /plan --- The ``Big Brain'' Phase}
    This is where we pretend we're doing architecture diagrams on purpose:
    
    \begin{itemize}
      \item \textbf{Architecture} \\
            What the system should look like, not whatever the AI hallucinated at 2 a.m.
      \item \textbf{Tech stack decisions} \\
            Picking tools intentionally --- not because they appeared in a YouTube thumbnail.
      \item \textbf{Data flow} \\
            How information is \emph{supposed} to move... instead of whatever mysterious path it took last time.
      \item \textbf{Constraints \& options} \\
            Performance, deployment, security --- the adult stuff.
    \end{itemize}
    
    \bigskip
    \textbf{/plan is the AI's blueprint. Without it, the AI builds you a treehouse when you asked for a garage.}
    \end{frame}
    

% ------------------------------------------------
\begin{frame}{Step 3: /tasks --- Tiny Steps, Huge Payoff}
    Tasks exist to prevent chaos and emotional damage:
    
    \begin{itemize}
      \item \textbf{Atomic, buildable steps} \\
            If a task takes more than one breath to read, break it up.
      \item \textbf{Dependency order} \\
            Because ``surprise prerequisites'' are how projects die.
      \item \textbf{Status tracking} \\
            (Todo / In Progress / Done / ``Why did Past-Me do this?'')
      \item \textbf{A roadmap the AI can actually follow} \\
            Think turn-by-turn GPS\ldots{} not ``good luck, buddy.''
    \end{itemize}
    
    \bigskip
    \textbf{If a task is unclear, the AI will \emph{absolutely} produce unclear code — and it will do so with confidence.}
    \end{frame}
    

% ------------------------------------------------
\begin{frame}{Step 4: /implement --- Where We Let the AI Cook}
    Where the ``magic'' happens\ldots{} 
    (and by magic, we mean carefully supervised automation):
    
    \begin{itemize}
      \item \textbf{Trigger tasks one at a time} \\
            Giving the AI \emph{all} the tasks at once is how horror movies start.
      \item \textbf{AI generates code guided by spec + plan} \\
            Finally using the instructions we so lovingly crafted.
      \item \textbf{Inspect results} \\
            Trust is great\ldots{} but verification prevents therapy bills.
      \item \textbf{Iterate cleanly} \\
            Fix, refine, repeat --- not ``burn it down and start over.''
    \end{itemize}
    
    \bigskip
    \textbf{A simple loop: Specify $\rightarrow$ Plan $\rightarrow$ Tasks $\rightarrow$ Implement.}\\
    Rinse. Repeat. Minimal screaming.
    \end{frame}
    

% ------------------------------------------------
\section{Conclusion}

\begin{frame}{The Takeaway}
Spec-Kit transforms AI development by:

\begin{itemize}
  \item \textbf{Providing clarity and structure} \\
        Because ``guess what I meant'' is not a development strategy.
  \item \textbf{Enabling repeatable workflows} \\
        You know\ldots{} like real engineering.
  \item \textbf{Reducing rework and hallucinated code} \\
        The AI will still hallucinate --- just less, and more politely.
  \item \textbf{Making AI coding assistants actually useful} \\
        Instead of creative writers accidentally doing software engineering.
\end{itemize}

\bigskip
\textbf{It's time to retire ``prompt and pray.'' We're doing actual engineering now.}
\end{frame}


% ------------------------------------------------
\begin{frame}
    \centering
  
    {\Huge Thank you!}
  
    \bigskip
  
    {\large You've been an amazing audience}\\
    
    {\small --- significantly less chaotic than my AI agents.}
  
    \vfill
  
    {\Large Questions?}
  
    \medskip
  
    {\small (Preferably about Spec-Kit\ldots{} but I am also prepared for otter-related inquiries.)}
  
    \bigskip
  
    {\tiny If your question is ``Why did the AI do that?'', the answer is: because you trusted it.}
  \end{frame}
  

% ------------------------------------------------
\end{document}
